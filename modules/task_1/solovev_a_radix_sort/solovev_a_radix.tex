\documentclass{report}

\usepackage[T2A]{fontenc}
\usepackage[utf8]{luainputenc}
\usepackage[english, russian]{babel}
\usepackage[pdftex]{hyperref}
\usepackage[14pt]{extsizes}
\usepackage{listings}
\usepackage{color}
\usepackage{geometry}
\usepackage{enumitem}
\usepackage{multirow}
\usepackage{graphicx}
\usepackage{indentfirst}
\usepackage{amsmath}


\geometry{a4paper,top=2cm,bottom=3cm,left=2cm,right=1.5cm}
\setlength{\parskip}{0.5cm}
\setlist{nolistsep, itemsep=0.3cm,parsep=0pt}

\lstset{language=C++,
		basicstyle=\footnotesize,
		keywordstyle=\color{blue}\ttfamily,
		stringstyle=\color{red}\ttfamily,
		commentstyle=\color{green}\ttfamily,
		morecomment=[l][\color{magenta}]{\#}, 
		tabsize=4,
		breaklines=true,
  		breakatwhitespace=true,
  		title=\lstname,       
}

\makeatletter
\renewcommand\@biblabel[1]{#1.\hfil}
\makeatother

\begin{document}

\begin{titlepage}

\begin{center}
Министерство науки и высшего образования Российской Федерации
\end{center}

\begin{center}
Федеральное государственное автономное образовательное учреждение высшего образования \\
Национальный исследовательский Нижегородский государственный университет им. Н.И. Лобачевского
\end{center}

\begin{center}
Институт информационных технологий, математики и механики
\end{center}

\vspace{4em}

\begin{center}
\textbf{\LargeОтчет по лабораторной работе} \\
\end{center}
\begin{center}
\textbf{\Large«Поразрядная сортировка целых чисел с простым слиянием»} \\
\end{center}

\vspace{4em}

\newbox{\lbox}
\savebox{\lbox}{\hbox{text}}
\newlength{\maxl}
\setlength{\maxl}{\wd\lbox}
\hfill\parbox{7cm}{
\hspace*{5cm}\hspace*{-5cm}\textbf{Выполнил:} \\ студент группы 381806-1 \\ Соловьев А. Ю.\\
\\
\hspace*{5cm}\hspace*{-5cm}\textbf{Проверил:}\\ доцент кафедры МОСТ, \\ кандидат технических наук \\ Сысоев А. В.\\
}
\vspace{\fill}

\begin{center} Нижний Новгород \\ 2021 \end{center}

\end{titlepage}

\setcounter{page}{2}

% Содержание
\tableofcontents
\newpage

% Введение
\section*{Введение}
\addcontentsline{toc}{section}{Введение}
Людям, как и компьютерам удобнее работать с упорядоченными наборами данных. Основным способом упорядочения множества элементов является сортировка.
\par Очень часто сортировки представляют собой алгоритмы, требующие многократных операций сравнения и обходов сортируемого диапазона. Логично предположить, что чем больше количество элементов, которые необходимо отсортировать, тем больше времени требуется для сортировки. Уменьшить затраты времени на сортировку может помочь распараллеливание реализации алгоритма с помощью его запуска на нескольких процессорах вычислительного оборудования.
\par Следовательно, мы можем разделить исходный набор данных на части, для каждой вычислительной единицы поровну, отсортировать их независимо друг от друга, а затем объединить, чтобы получить отсортированный набор.
\par В ходе данной лабораторной работе будет реализован один из алгоритмов сортировок - поразрядная сортировка, а также алгоритм простого слияния.

\newpage

% Постановка задачи
\section*{Постановка задачи}
\addcontentsline{toc}{section}{Постановка задачи}
В рамках лабораторной работы необходимо реализовать последовательную и параллельную реализации алгоритма поразрядной сортировки для целых чисел с простым слиянием, проверить корректность работы алгоритмов, сравнить эффективность последовательной и параллельной реализаций.
\par Для реализации параллельной версии необходимо использовать библиотеки OpenMP и TBB. Для проверки корректности работы алгоритмов используется Google Testing Framework.
\newpage

% Метод решения
\section*{Описание алгоритмов}
\addcontentsline{toc}{section}{Метод решения}
Поразрядная сортировка позволяет отсортировать исходный массив данных, в моем случае, мы сортируем только целые числа, но в принципе, мы можем сортировать любые типы данных. 
\par Шаги алгоритма:
\begin{enumerate}
\item В начале алгоритма разобьем наши элементы на положительные и отрицательные, и сохраним в двух разных массивах. Это сделано, для корректного поразрядного сравнения.
\item Отдельные массивы положительных и отрицательных чисел будем сортировать. Сравнение производится поразрядно, сначала сравниваются значения одного, крайнего разряда, затем элементы группируются, и затем сравниваются следующие разряды.
\end{enumerate}
\par Сложность алгоритма варьируется от $ O(n*k)$ до $O(n)$, где $n$ --- количество элементов в массиве.

\newpage

% Схема распараллеливания
\section*{Схема распараллеливания}
\addcontentsline{toc}{section}{Схема распараллеливания}
Для распараллеливания алгоритма поразрядной сортировки необходимо разделить массив на равные по размеру фрагменты, количество которых равно количеству потоков. Если размер исходного массива не делится нацело на количество потоков, необходимо работу с остаточными элементами производить на последнем по счету потоке. Затем эти фрагменты сортируются параллельно на разных потоках. Далее выполняется простое слияние отсортированных фрагментов массива на одном потоке.

\newpage

% Описание программной реализации
\section*{Описание программной реализации}
\addcontentsline{toc}{section}{Описание программной реализации}
Алгоритм сортировки Хоара вызывается с помощью функции:
\begin{lstlisting}
int RadixSort(std::vector<int> *buffer);
\end{lstlisting}
\par Входными параметрами являются: исходный массив
\par Реализация распараллеливания сортировки представлена в функциях:
\begin{lstlisting}
ParallelSorting(std::vector<int> *buffer);
\end{lstlisting}
\begin{lstlisting}
int ParallelSortingTBB(std::vector<int> *arr);
\end{lstlisting}
\par На вход они принимают только исходный массив.
\par Слияние осуществляется за счет функций:
\begin{lstlisting}
int MergeArrays(std::vector<int> *buffer1, std::vector<int> *buffer2, int left, std::vector<int> *result);
\end{lstlisting}
\par Входными параметрами являются: массивы отрицательных и положительных чисел, а также результирующий вектор.
\begin{lstlisting}
void merge(int* a, int size_a, int* b, int size_b)
\end{lstlisting}
\par На вход она принимает два массива и их размеры.
\newpage

% Подтверждение корректности
\section*{Подтверждение корректности}
\addcontentsline{toc}{section}{Подтверждение корректности}
Для подтверждения корректности в программе представлен набор тестов, разработанных с Google Testing Framework.
\par Набор представляет из себя тесты, которые проверяют корректность вычислений (сравниваются массивы, полученные из последовательного и параллельного алгоритма), а также эффективность (вычисление занимаемого последовательной и параллельной сортировок времени и сравнение полученных данных).
\par Успешное прохождение всех тестов подтверждает корректность работы написанной программы.
\newpage

% Результаты экспериментов
\section*{Результаты экспериментов}
\addcontentsline{toc}{section}{Результаты экспериментов}
Вычислительные эксперименты для оценки эффективности параллельного алгоритма сортировки Хоара проводились на оборудовании со следующей аппаратной конфигурацией:

\begin{itemize}
\item Процессор: Intel Core i5-3470f, 2.40 GHz, 4 ядра;
\item Оперативная память: 8 ГБ;
\item ОС: Microsoft Windows 10 Home 64-bit.
\end{itemize}

\par В рамках эксперимента было вычислено время работы последовательного и параллельного алгоритма поразрядной сортировки.
\par В качестве тестируемых данных генерируется массив случайных чисел типа double от -10000 до 10000. Размер массива --- 150000000 элементов.
\par Результаты экспериментов представлены ниже.

\begin{table}[!h]
\caption{Результаты экспериментов версии с OpenMP}
\centering
\begin{tabular}{lllll}
Число потоков & Последовательно (OpenMP) & Параллельно & Ускорение  \\
4        & 12.23412         & 3.81234     & 3.2     \\
      
\end{tabular}
\end{table}

\begin{table}[!h]
\caption{Результаты экспериментов версии с TBB}
\centering
\begin{tabular}{lllll}
Число потоков & Последовательно (TBB) & Параллельно & Ускорение  \\
4        & 12.23412         & 4.35312    & 2.8       \\
      
\end{tabular}
\end{table}

\par По данным экспериментов видно, что параллельный алгоритм поразрядной сортировки работает значительно быстрее, но из-за примущества последовательной реализации, в виде не слияния, а скоейки отрицательной и положительной части исходного вектора, ускорение не идеально.
\newpage

% Заключение
\section*{Заключение}
\addcontentsline{toc}{section}{Заключение}
В ходе выполнения лабораторной работы был разработан последовательный и два параллельных алгоритма поразрядной сортировки целых чисел с простым слиянием.
\par По данным экспериментов удалось сравнить время работы последовательной и параллельных реализаций. Выявлено, что параллельные алгоритмы работают в несколько раз быстрее, что подтверждает преимущество параллельных вычислений.
\par Для подтверждения корректности работы программы написаны тесты с использованием Google Testing Framework.
\newpage

% Список литературы
\begin{thebibliography}{1}
\addcontentsline{toc}{section}{Список литературы}
\bibitem{wiki} Википедия: Быстрая сортировка  \url {https://en.wikipedia.org/wiki/Radix_sort} 
\bibitem{wiki1} Википедия: Сортировка слиянием \url {https://en.wikipedia.org/wiki/Merge_sort}
\end{thebibliography}
\newpage

% Приложение
\section*{Приложение}
\addcontentsline{toc}{section}{Приложение}
В этом разделе находится листинг всего кода, написанного в рамках лабораторной работы.
\par Реализация с использованием технологии OpenMP:
\begin{lstlisting}
// Copyright 2021 Solovev Aleksandr
#ifndef MODULES_TASK_2_SOLOVEV_A_RADIX_SORT_RADIX_SORT_H_
#define MODULES_TASK_2_SOLOVEV_A_RADIX_SORT_RADIX_SORT_H_
#include <omp.h>
#include <time.h>
#include <stdlib.h>
#include <random>
#include <vector>

int generateRandomArray(std::vector<int> *buffer, int min, int max);
int SortingCheck(std::vector<int> *buffer);
int RadixSort(std::vector<int> *buffer);
int ParallelSorting(std::vector<int> *buffer);
#endif  // MODULES_TASK_2_SOLOVEV_A_RADIX_SORT_RADIX_SORT_H_
\end{lstlisting}
\begin{lstlisting}
// Copyright 2021 Solovev Aleksandr
#include "../../../modules/task_2/solovev_a_radix_sort/radix_sort.h"
#include <stdio.h>
#include <cstring>
#include <utility>




int generateRandomArray(std::vector<int> *buffer, int min, int max) {
    std::mt19937 gen;
    gen.seed(static_cast<unsigned int>(time(0)));
    for (size_t i = 0; i < buffer->size(); i++) {
        buffer->at(i) = static_cast<int>(gen()) % (max - min + 1) + min;
    }
    return 0;
}

int MergeArrays(std::vector<int> *buffer1, std::vector<int> *buffer2, int left, std::vector<int> *result) {
    int current_buff = 0;
    for (size_t i = left; i < buffer1->size() + left; i++) {
        result->at(i) = buffer1->at(current_buff);
        current_buff++;
    }
    current_buff = 0;
    for (size_t i = left + buffer1->size() ; i < buffer2->size() + left + buffer1->size(); i++) {
        result->at(i) = buffer2->at(current_buff);
        current_buff++;
    }

    return 0;
}



int SortingCheck(std::vector<int> *buffer) {
    for (size_t i = 0; i < buffer->size() - 1; i++) {
        if (buffer->at(i) <= buffer->at(i + 1)) {
            continue;
        } else {
            return -1;
        }
    }
    return 0;
}

void CountingSort(std::vector<int> *input, std::vector<int> *output, int valbyte) {
    unsigned char *buffer = (unsigned char *)input->data();
    int counter[256] = {0};
    size_t length_counter = 256;
    int value = 0;

    for (size_t i = 0; i < input->size(); i++) {
        counter[buffer[4 * i + valbyte]]++;
    }

    for (size_t j = 0; j < length_counter; j++) {
        int tmp = counter[j];
        counter[j] = value;
        value += tmp;
    }

    for (size_t i = 0; i < input->size(); i++) {
        output->at(counter[buffer[4 * i + valbyte]]++) = input->at(i);
    }
}

int RadixSortUnsigned(std::vector<int> *buffer) {
    if (buffer->empty()) {
        return -1;
    }
    std::vector<int> outbuf(buffer->size());
    if (outbuf.data() == nullptr) {
        return -1;
    }
    for (int i = 0; i < 2; i++) {
        CountingSort(buffer, &outbuf, 2 * i);
        CountingSort(&outbuf, buffer, 2 * i + 1);
    }
    return 0;
}
int RadixSortUnsignedParallel(std::vector<int> *buffer) {
    if (buffer->empty()) {
        return -1;
    }
    std::vector<int> outbuf(buffer->size());
    if (outbuf.data() == nullptr) {
        return -1;
    }
    for (int i = 0; i < 2; i++) {
        CountingSort(buffer, &outbuf, 2 * i);
        CountingSort(&outbuf, buffer, 2 * i + 1);
    }
    return 0;
}
int RadixSort(std::vector<int> *buffer) {
    if (buffer->size() < 1)
        return -1;
    int positive_length = 0, negative_length = 0;
    int status = 0;

    for (size_t i = 0; i < buffer->size(); i++) {
        if (buffer->at(i) >= 0) {
            positive_length++;
        } else {
            negative_length++;
        }
    }
    std::vector<int> positive_numbers(positive_length);
    std::vector<int> negative_numbers(negative_length);
    positive_length = 0;
    negative_length = 0;
    for (size_t i = 0; i < buffer->size(); i++) {
        if (buffer->at(i) >= 0) {
            positive_numbers[positive_length] = buffer->at(i);
            positive_length++;
        } else {
            negative_numbers[negative_length] = buffer->at(i);
            negative_length++;
        }
    }

    status = RadixSortUnsigned(&positive_numbers);
    status = RadixSortUnsigned(&negative_numbers);

    status = MergeArrays(&negative_numbers,  &positive_numbers, 0,  buffer);
    return status;
}
int RadixSortParallel(std::vector<int> *buffer, int left, int right) {
    if (buffer->size() < 1)
        return -1;
    int positive_length = 0, negative_length = 0;
    int status = 0;
    for (int i = left; i <= right; i++) {
        if (buffer->at(i) >= 0) {
            positive_length+=1;
        } else {
            negative_length+=1;
        }
    }
    std::vector<int> positive_numbers(positive_length);
    std::vector<int> negative_numbers(negative_length);
    positive_length = 0;
    negative_length = 0;

    for (int i = left; i <= right; i++) {
        if (buffer->at(i) >= 0) {
            positive_numbers[positive_length] = buffer->at(i);
            positive_length+=1;
        } else {
            negative_numbers[negative_length] = buffer->at(i);
            negative_length+=1;
        }
    }

    status = RadixSortUnsigned(&positive_numbers);
    status = RadixSortUnsigned(&negative_numbers);

    status = MergeArrays(&negative_numbers,  &positive_numbers, left,  buffer);

    return status;
}
void merge(int* a, int size_a, int* b, int size_b) {
    int i = 0, j = 0, k = 0;
    int size_c = size_a + size_b;
    int* c = new int[size_c];
    while ((i < size_a) && (j < size_b)) {
        if (a[i] <= b[j]) {
            c[k++] = a[i++];
        } else {
            c[k++] = b[j++];
        }
    }

    if (i == size_a) {
        while (j < size_b) {
            c[k++] = b[j++];
        }
    } else {
        while (i < size_a) {
            c[k++] = a[i++];
        }
    }

    for (int i = 0; i < size_c; i++) {
        a[i] = c[i];
    }
}

int ParallelSorting(std::vector<int> *buffer) {
    int part_vec_size = 0;
    int n_threads = 0;
    #pragma omp parallel
    {
        n_threads = omp_get_num_threads();
        part_vec_size = static_cast<int>(buffer->size()) / n_threads;
        int thread_id = omp_get_thread_num();
        if (thread_id != n_threads - 1) {
            RadixSortParallel(buffer, thread_id * part_vec_size, (thread_id + 1) * part_vec_size - 1);
        } else {
            RadixSortParallel(buffer, thread_id * part_vec_size, static_cast<int>(buffer->size()) - 1);
        }
    }
    for (int i = 1; i < n_threads; i++) {
        int current_size = part_vec_size;
        if (i == n_threads - 1) {
            current_size = static_cast<int>(buffer->size()) - (n_threads - 1) * part_vec_size;
        }
        merge(buffer->data(), part_vec_size * i, buffer->data() + part_vec_size * i, current_size);
    }
    return 0;
}
\end{lstlisting}
\begin{lstlisting}
// Copyright 2021 Solovev Aleksandr
#include <numeric>
#include <utility>
#include <algorithm>
#include "gtest/gtest.h"
#include "../../../modules/task_2/solovev_a_radix_sort/radix_sort.h"


TEST(Parallel, Test_Very_Micro_Length) {
    const int length = 10;
    int statusseq = 0;
    int statuspar = 0;
    std::vector<int> buffer1(length);
    statusseq = generateRandomArray(&buffer1, -100, 100);
    std::vector<int> buffer2 = buffer1;
    statusseq = RadixSort(&buffer1);
    statusseq = SortingCheck(&buffer1);
    statuspar = ParallelSorting(&buffer2);
    statuspar = SortingCheck(&buffer2);

    ASSERT_EQ(statusseq, statuspar);
}


TEST(Parallel, Test_Very_Small_Length) {
    const int length = 100;
    int statusseq = 0;
    int statuspar = 0;
    std::vector<int> buffer1(length);
    statusseq = generateRandomArray(&buffer1, -1000, 1000);
    std::vector<int> buffer2 = buffer1;
    statusseq = RadixSort(&buffer1);
    statusseq = SortingCheck(&buffer1);
    statuspar = ParallelSorting(&buffer2);
    statuspar = SortingCheck(&buffer2);

    ASSERT_EQ(statusseq, statuspar);
}

TEST(Parallel, Test_Small_length) {
    const int length = 500;
    int statusseq = 0;
    int statuspar = 0;
    std::vector<int> buffer1(length);
    statusseq = generateRandomArray(&buffer1, -10000, 10000);

    std::vector<int> buffer2 = buffer1;
    statusseq = RadixSort(&buffer1);
    statusseq = SortingCheck(&buffer1);
    statuspar = ParallelSorting(&buffer2);
    statuspar = SortingCheck(&buffer2);

    ASSERT_EQ(statusseq, statuspar);
}

TEST(Parallel, Test_Medium_Length) {
    const int length = 1000;
    int statusseq = 0;
    int statuspar = 0;
    std::vector<int> buffer1(length);
    statusseq = generateRandomArray(&buffer1,  -1000, 1000);
    std::vector<int> buffer2 = buffer1;
    statusseq = RadixSort(&buffer1);
    statusseq = SortingCheck(&buffer1);

    statuspar = ParallelSorting(&buffer2);
    statuspar = SortingCheck(&buffer2);


    ASSERT_EQ(statusseq, statuspar);
}

TEST(Senquential, Test_Large_Length) {
    const int length = 1500;
    int statusseq = 0;
    int statuspar = 0;
    std::vector<int> buffer1(length);
    statusseq = generateRandomArray(&buffer1,  -1000, 1000);
    std::vector<int> buffer2 = buffer1;

    statusseq = RadixSort(&buffer1);
    statusseq = SortingCheck(&buffer1);

    statuspar = ParallelSorting(&buffer2);
    statuspar = SortingCheck(&buffer2);


    ASSERT_EQ(statusseq, statuspar);
}

TEST(Senquential, Test_Very_Large_Length) {
    const int length = 2000;
    int statusseq = 0;
    int statuspar = 0;
    std::vector<int> buffer1(length);
    statusseq = generateRandomArray(&buffer1,  -5000, 5000);
    std::vector<int> buffer2 = buffer1;
    statusseq = RadixSort(&buffer1);
    statusseq = SortingCheck(&buffer1);

    statuspar = ParallelSorting(&buffer2);
    statuspar = SortingCheck(&buffer2);


    ASSERT_EQ(statusseq, statuspar);
}

// TEST(Seqvssspar, Test_Only_Positive) {
//     const int length = 150000000;
//     int statusseq = 0;
//     int statuspar = 0;
//     std::vector<int> buffer1(length);
//     statusseq = generateRandomArray(&buffer1, -10000, 10000);
//     std::vector<int> buffer2 = buffer1;
//     double t1_seq = omp_get_wtime();
//     statusseq = RadixSort(&buffer1);
//     statusseq = SortingCheck(&buffer1);
//     double t2_seq = omp_get_wtime();
//     double time_seq = t2_seq - t1_seq;
//     std::cout << "Time seq: " << time_seq << std::endl;
//     double t1_omp = omp_get_wtime();
//     statuspar = ParallelSorting(&buffer2);
//     statuspar = SortingCheck(&buffer2);
//     double t2_omp = omp_get_wtime();
//     double time_omp = t2_omp - t1_omp;
//     std::cout << "Time omp: " << time_omp << std::endl;
//     std::cout << "Acceleration: " << time_seq / time_omp << std::endl;
//     ASSERT_EQ(buffer2, buffer1);
// }

int main(int argc, char** argv) {
    ::testing::InitGoogleTest(&argc, argv);
    return RUN_ALL_TESTS();
}

\end{lstlisting}

\par Реализация с использованием технологии TBB:

\begin{lstlisting}
// Copyright 2021 Solovev Aleksandr
#ifndef MODULES_TASK_3_SOLOVEV_A_RADIX_SORT_RADIX_SORT_H_
#define MODULES_TASK_3_SOLOVEV_A_RADIX_SORT_RADIX_SORT_H_

#include <time.h>
#include <vector>
#include <random>

int generateRandomArray(std::vector<int> *buffer, int min, int max);
int SortingCheck(std::vector<int> *buffer);
int RadixSort(std::vector<int> *buffer);
int ParallelSortingTBB(std::vector<int> *buffer);

#endif  // MODULES_TASK_3_SOLOVEV_A_RADIX_SORT_RADIX_SORT_H_

\end{lstlisting}
\begin{lstlisting}
// Copyright 2021 Solovev Aleksandr

#include <tbb/blocked_range.h>
#include <tbb/parallel_for.h>

#include "../../../modules/task_3/solovev_a_radix_sort/radix_sort.h"

int generateRandomArray(std::vector<int> *buffer, int min, int max) {
    std::mt19937 gen;
    gen.seed(static_cast<unsigned int>(time(0)));
    for (size_t i = 0; i < buffer->size(); i++) {
        buffer->at(i) = static_cast<int>(gen()) % (max - min + 1) + min;
    }
    return 0;
}

int MergeArrays(std::vector<int> *buffer1, std::vector<int> *buffer2, int left, std::vector<int> *result) {
    int current_buff = 0;
    for (size_t i = left; i < buffer1->size() + left; i++) {
        result->at(i) = buffer1->at(current_buff);
        current_buff++;
    }
    current_buff = 0;
    for (size_t i = left + buffer1->size() ; i < buffer2->size() + left + buffer1->size(); i++) {
        result->at(i) = buffer2->at(current_buff);
        current_buff++;
    }

    return 0;
}



int SortingCheck(std::vector<int> *buffer) {
    for (size_t i = 0; i < buffer->size() - 1; i++) {
        if (buffer->at(i) <= buffer->at(i + 1)) {
            continue;
        } else {
            return -1;
        }
    }
    return 0;
}

void CountingSort(std::vector<int> *input, std::vector<int> *output, int valbyte) {
    unsigned char *buffer = (unsigned char *)input->data();
    int counter[256] = {0};
    size_t length_counter = 256;
    int value = 0;

    for (size_t i = 0; i < input->size(); i++) {
        counter[buffer[4 * i + valbyte]]++;
    }

    for (size_t j = 0; j < length_counter; j++) {
        int tmp = counter[j];
        counter[j] = value;
        value += tmp;
    }

    for (size_t i = 0; i < input->size(); i++) {
        output->at(counter[buffer[4 * i + valbyte]]++) = input->at(i);
    }
}

int RadixSortUnsigned(std::vector<int> *buffer) {
    if (buffer->empty()) {
        return -1;
    }
    std::vector<int> outbuf(buffer->size());
    if (outbuf.data() == nullptr) {
        return -1;
    }
    for (int i = 0; i < 2; i++) {
        CountingSort(buffer, &outbuf, 2 * i);
        CountingSort(&outbuf, buffer, 2 * i + 1);
    }
    return 0;
}
int RadixSort(std::vector<int> *buffer) {
    if (buffer->size() < 1)
        return -1;
    int positive_length = 0, negative_length = 0;
    int status = 0;

    for (size_t i = 0; i < buffer->size(); i++) {
        if (buffer->at(i) >= 0) {
            positive_length++;
        } else {
            negative_length++;
        }
    }
    std::vector<int> positive_numbers(positive_length);
    std::vector<int> negative_numbers(negative_length);
    positive_length = 0;
    negative_length = 0;
    for (size_t i = 0; i < buffer->size(); i++) {
        if (buffer->at(i) >= 0) {
            positive_numbers[positive_length] = buffer->at(i);
            positive_length++;
        } else {
            negative_numbers[negative_length] = buffer->at(i);
            negative_length++;
        }
    }

    status = RadixSortUnsigned(&positive_numbers);
    status = RadixSortUnsigned(&negative_numbers);

    status = MergeArrays(&negative_numbers,  &positive_numbers, 0,  buffer);
    return status;
}
int RadixSortParallel(std::vector<int> *buffer, int left, int right) {
    if (buffer->size() < 1)
        return -1;
    int positive_length = 0, negative_length = 0;
    int status = 0;
    for (int i = left; i <= right; i++) {
        if (buffer->at(i) >= 0) {
            positive_length+=1;
        } else {
            negative_length+=1;
        }
    }
    std::vector<int> positive_numbers(positive_length);
    std::vector<int> negative_numbers(negative_length);
    positive_length = 0;
    negative_length = 0;

    for (int i = left; i <= right; i++) {
        if (buffer->at(i) >= 0) {
            positive_numbers[positive_length] = buffer->at(i);
            positive_length+=1;
        } else {
            negative_numbers[negative_length] = buffer->at(i);
            negative_length+=1;
        }
    }

    status = RadixSortUnsigned(&positive_numbers);
    status = RadixSortUnsigned(&negative_numbers);

    status = MergeArrays(&negative_numbers,  &positive_numbers, left,  buffer);

    return status;
}
void merge(int* a, int size_a, int* b, int size_b) {
    int i = 0, j = 0, k = 0;
    int size_c = size_a + size_b;
    int* c = new int[size_c];
    while ((i < size_a) && (j < size_b)) {
        if (a[i] <= b[j]) {
            c[k++] = a[i++];
        } else {
            c[k++] = b[j++];
        }
    }

    if (i == size_a) {
        while (j < size_b) {
            c[k++] = b[j++];
        }
    } else {
        while (i < size_a) {
            c[k++] = a[i++];
        }
    }

    for (int i = 0; i < size_c; i++) {
        a[i] = c[i];
    }
}

int ParallelSortingTBB(std::vector<int> *arr) {
    int size = static_cast<int>(arr->size());
    int min_grain_size = 10;
    int divider = 5;
    int part_arr_size = size / divider;
    int grain_size = part_arr_size > min_grain_size ? part_arr_size : min_grain_size;
    tbb::parallel_for(tbb::blocked_range<int>(0, size, grain_size), [&arr](const tbb::blocked_range<int>& r) {
        RadixSortParallel(arr, r.begin(), r.end() - 1);
    });
    int i_prev = 0;
    int k = 0;
    for (int i = 1; i < size; i++) {
        if (arr->at(i) < arr->at(i - 1)) {
            if (k++ > 0)
                merge(arr->data(), i_prev, arr->data() + i_prev, i - i_prev);
            i_prev = i;
        }
    }
    merge(arr->data(), i_prev, arr->data() + i_prev, size - i_prev);
    return 0;
}

\end{lstlisting}
\begin{lstlisting}
// Copyright 2021 Solovev Aleksandr
#include <gtest/gtest.h>
#include <tbb/tick_count.h>

#include "../../../modules/task_3/solovev_a_radix_sort/radix_sort.h"


TEST(Parallel, Test_Very_Micro_Length) {
    const int length = 10;
    int statusseq = 0;
    int statuspar = 0;
    std::vector<int> buffer1(length);
    statusseq = generateRandomArray(&buffer1, -100, 100);
    std::vector<int> buffer2 = buffer1;
    statusseq = RadixSort(&buffer1);
    statusseq = SortingCheck(&buffer1);
    statuspar = ParallelSortingTBB(&buffer2);
    statuspar = SortingCheck(&buffer2);

    ASSERT_EQ(statusseq, statuspar);
}


TEST(Parallel, Test_Very_Small_Length) {
    const int length = 100;
    int statusseq = 0;
    int statuspar = 0;
    std::vector<int> buffer1(length);
    statusseq = generateRandomArray(&buffer1, -1000, 1000);
    std::vector<int> buffer2 = buffer1;
    statusseq = RadixSort(&buffer1);
    statusseq = SortingCheck(&buffer1);
    statuspar = ParallelSortingTBB(&buffer2);
    statuspar = SortingCheck(&buffer2);

    ASSERT_EQ(statusseq, statuspar);
}

TEST(Parallel, Test_Small_length) {
    const int length = 500;
    int statusseq = 0;
    int statuspar = 0;
    std::vector<int> buffer1(length);
    statusseq = generateRandomArray(&buffer1, -10000, 10000);

    std::vector<int> buffer2 = buffer1;
    statusseq = RadixSort(&buffer1);
    statusseq = SortingCheck(&buffer1);
    statuspar = ParallelSortingTBB(&buffer2);
    statuspar = SortingCheck(&buffer2);

    ASSERT_EQ(statusseq, statuspar);
}

TEST(Parallel, Test_Medium_Length) {
    const int length = 1000;
    int statusseq = 0;
    int statuspar = 0;
    std::vector<int> buffer1(length);
    statusseq = generateRandomArray(&buffer1,  -1000, 1000);
    std::vector<int> buffer2 = buffer1;
    statusseq = RadixSort(&buffer1);
    statusseq = SortingCheck(&buffer1);

    statuspar = ParallelSortingTBB(&buffer2);
    statuspar = SortingCheck(&buffer2);


    ASSERT_EQ(statusseq, statuspar);
}

TEST(Senquential, Test_Large_Length) {
    const int length = 1500;
    int statusseq = 0;
    int statuspar = 0;
    std::vector<int> buffer1(length);
    statusseq = generateRandomArray(&buffer1,  -1000, 1000);
    std::vector<int> buffer2 = buffer1;

    statusseq = RadixSort(&buffer1);
    statusseq = SortingCheck(&buffer1);

    statuspar = ParallelSortingTBB(&buffer2);
    statuspar = SortingCheck(&buffer2);


    ASSERT_EQ(statusseq, statuspar);
}

TEST(Senquential, Test_Very_Large_Length) {
    const int length = 2000;
    int statusseq = 0;
    int statuspar = 0;
    std::vector<int> buffer1(length);
    statusseq = generateRandomArray(&buffer1,  -5000, 5000);
    std::vector<int> buffer2 = buffer1;
    statusseq = RadixSort(&buffer1);
    statusseq = SortingCheck(&buffer1);

    statuspar = ParallelSortingTBB(&buffer2);
    statuspar = SortingCheck(&buffer2);


    ASSERT_EQ(statusseq, statuspar);
}

// TEST(Seqvssspar, Test_Only_Positive) {
    // int n = 3000000;
    // double start = 0, finish = 0, time_seq = 0, time_par = 0;
    // std::vector<double> vec = createRandomVector(n);
    // std::vector<double> vec_copy = vec;
    // tbb::tick_count t1_seq = tbb::tick_count::now();
    // qSortSeq(&vec, 0, n - 1);
    // tbb::tick_count t2_seq = tbb::tick_count::now();
    // time_seq = (t2_seq - t1_seq).seconds();
    // std::cout << "Time sequential: " << time_seq << std::endl;
    // tbb::tick_count t1_par = tbb::tick_count::now();
    // qSortTbb(&vec_copy);
    // tbb::tick_count t2_par = tbb::tick_count::now();
    // time_par = (t2_par - t1_par).seconds();
    // std::cout << "Time parallel: " << time_par << std::endl;
    // std::cout << "Acceleration " << time_seq / time_par << std::endl;
//     const int length = 15000000;
//     int statusseq = 0;
//     int statuspar = 0;
//     std::vector<int> buffer1(length);
//     statusseq = generateRandomArray(&buffer1, -10000, 10000);
//     std::vector<int> buffer2 = buffer1;
//     tbb::tick_count t1_seq = tbb::tick_count::now();
//     statusseq = RadixSort(&buffer1);
//     statusseq = SortingCheck(&buffer1);
//     tbb::tick_count t2_seq = tbb::tick_count::now();
//     double time_seq = (t2_seq - t1_seq).seconds();
//     std::cout << "Time sequential: " << time_seq << std::endl;
//     tbb::tick_count t1_par = tbb::tick_count::now();
//     statuspar = ParallelSortingTBB(&buffer2);
//     tbb::tick_count t2_par = tbb::tick_count::now();
//     double time_par = (t2_par - t1_par).seconds();
//     std::cout << "Time parallel: " << time_par << std::endl;
//     std::cout << "Acceleration " << time_seq / time_par << std::endl;
//     ASSERT_EQ(buffer2, buffer1);
// }

int main(int argc, char** argv) {
    ::testing::InitGoogleTest(&argc, argv);
    return RUN_ALL_TESTS();
}


\end{lstlisting}

\end{document}